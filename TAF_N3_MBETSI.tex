\documentclass[11pt,a4paper]{article}
\usepackage[utf8]{inputenc}
\usepackage[T1]{fontenc}
\usepackage[french]{babel}
\usepackage{xcolor}
\usepackage{graphicx}
\usepackage{fancyhdr}
\usepackage{hyperref}
\pagestyle{fancy}
\pagestyle{fancy}
\fancyhf{}  % vide l'en-tête et le pied

% --- Ligne d'en-tête et de pied ---
\renewcommand{\headrulewidth}{0pt}


% --- En-tête ---
\renewcommand{\sectionmark}[1]{\markboth{\thesection.\ #1}{}}
\fancyhead[R]{\makebox[\textwidth][r]{\textbf{Introduction aux Techniques de l'Investigation Numérique}}}

% --- Pied de page ---
\fancyfoot[L]{\textbf{Cybersécurité et Investigation Numérique -- IV}} % petit espace à gauche
\fancyfoot[C]{\hspace{11.5cm}\textbf{\thepage}}
\setlength{\headsep}{1.2cm}
% --- Ajuster l'espacement du pied de page pour éviter chevauchement ---
\setlength{\footskip}{1.5cm} 
\usepackage{geometry}
\usepackage{array}
\usepackage{ulem} % pour souligner plus joli
\usepackage{tcolorbox} % pour encadrer joliment
\geometry{margin=2cm}
\hypersetup{
	colorlinks = true,
	linkcolor = black
}


% Pour lignes horizontales doubles
\newcommand{\HRule}{\rule{\linewidth}{1pt}}
\newcommand{\DoubleHRule}{\rule{\linewidth}{1.5pt}\\[-0.3em]\rule{\linewidth}{0.8pt}}

\begin{document}
	\thispagestyle{empty} % pas de numérotation de page
	
	% --- Partie haute avec encadrement ---
	\begin{tcolorbox}[
		colback=blue!5,
		colframe=blue!50,
		boxrule=0.8pt,
		arc=4mm,
		left=4mm, right=4mm, top=2mm, bottom=2mm
		]
		
		\begin{tabular*}{\textwidth}{@{\extracolsep{\fill}} m{0.35\textwidth} m{0.25\textwidth} m{0.35\textwidth} }
			
			% Bloc gauche
			\centering
			\textbf{REPUBLIQUE DU CAMEROUN}\\
			Paix -- Travail -- Patrie\\
			\HRule \\[0.3em]
			\textbf{UNIVERSITE DE YAOUNDE I}\\
			\HRule \\[0.3em]
			\textbf{ECOLE NATIONALE SUPERIEURE\\POLYTECHNIQUE DE YAOUNDE}\\
			\HRule \\[0.3em]
			\textbf{DEPARTEMENT DE GENIE INFORMATIQUE}
			
			&
			% Logo au milieu
			\centering
			\includegraphics[width=3.5cm]{logo.png}
			
			&
			% Bloc droit
			\centering
			\textbf{REPUBLIC OF CAMEROON}\\
			Peace -- Work -- Fatherland\\
			\HRule \\[0.3em]
			\textbf{UNIVERSITY OF YAOUNDE I}\\
			\HRule \\[0.3em]
			\textbf{NATIONAL ADVANCED SCHOOL\\OF ENGINEERING OF YAOUNDE}\\
			\HRule \\[0.3em]
			\textbf{DEPARTMENT OF COMPUTER SCIENCE}
			
		\end{tabular*}
	\end{tcolorbox}
	
	\vspace{1.2cm}
	
	% --- Titre de l'exposé ---
	\begin{center}
		\DoubleHRule \\[1em]
		{\huge \bfseries Exercices du chapitre 2 }\\[1em]
		\DoubleHRule
	\end{center}
	
	\vspace{2cm}
	
	% --- Participants et superviseur ---
	\begin{tabular*}{\textwidth}{@{\extracolsep{\fill}} m{0.45\textwidth} m{0.45\textwidth} }
		\raggedright
		\uline{\textbf{Participant}} \\[1em]
		\textbf{Matricule :} 22P035 \\[0.8em]
		\textbf{Spécialité :} Cybersécurité et Investigation Numérique \\[0.8em]
		\textbf{Noms :} MBETSI DJOFANG AIME LINDSEY \\[0.8em]
		\textbf{Niveau :} 4
		&
		\raggedright
		\uline{\textbf{Superviseur}} \\[1em]
		M.\hspace{0.5cm} MINKA MI NGUIDJOI \\ 
		\hspace{1.1cm} Thierry Emmanuel
	\end{tabular*}
	
	\vfill
	
	% --- Pied de page ---
	\begin{center}
		\Large \textbf{Année Scolaire : 2025--2026}
	\end{center}
	
	\newpage
	\tableofcontents
	\newpage
	\setcounter{page}{3}
	
	
	
\newpage

\section{Partie 1 : Analyse Historique et Épistémologique}

\subsection*{1. Analyse Comparative des Régimes de Vérité}

\textbf{Périodes choisies :}
\begin{itemize}
	\item \textbf{1990-2000} : Ère de la Professionalisation
	\item \textbf{2010-2020} : Ère du Big Data et du Cloud
\end{itemize}

\textbf{Vecteurs de dominance :}
\begin{itemize}
	\item \(\vec{R}_{1990-2000} = (0.2, 0.4, 0.1, 0.3)\) 
	-- Dominance juridique (\(\alpha_J\)) et pratique (\(\alpha_P\))
	\item \(\vec{R}_{2010-2020} = (0.5, 0.2, 0.1, 0.2)\) 
	-- Dominance technique (\(\alpha_T\)) avec l'avènement du Big Data
\end{itemize}

\textbf{Discontinuités épistémologiques (Foucault) :}
\begin{itemize}
	\item Passage d'un régime où la vérité est validée par les tribunaux et la chaîne de custody à un régime où la vérité est produite par des algorithmes
	\item La preuve n'est plus humaine mais computationnelle
\end{itemize}

\textbf{Explication sociotechnique :}
\begin{itemize}
	\item \textbf{1990-2000} : Internet se démocratise → besoin de cadres juridiques et de standards professionnels
	\item \textbf{2010-2020} : Explosion des données → nécessité d'algorithmes pour analyser à grande échelle
\end{itemize}

\textbf{Transition :}
\begin{itemize}
	\item \textbf{Révolutionnaire} dans son impact, mais \textbf{progressive} dans son déploiement (accumulation technologique)
\end{itemize}

\subsection*{2. Étude de Cas Archéologique Foucaldienne}

\textbf{Affaire choisie : \textbf{Enron (2001)}}
\begin{itemize}
	\item \textbf{Formation discursive} : Discours sur la preuve électronique à grande échelle, l'analyse algorithmique, la standardisation
	\item \textbf{Dicible/pensable} : On peut parler de preuves automatisées ; on pense que la vérité peut émerger de l'analyse de masses de données
	\item \textbf{Régime de vérité} : Standardisation (2000-2010) → preuve légitime si conforme aux normes (ISO, NIST)
\end{itemize}

\textbf{Comparaison avec une affaire contemporaine (Silk Road, 2013) :}
\begin{itemize}
	\item \textbf{Régime computational} → preuve par analyse blockchain et couches techniques
	\item \textbf{Dicible/pensable} : On peut percer l'anonymat via les métadonnées et la blockchain
\end{itemize}

\section{Partie 2 : Modélisation Mathématique et Prospective}

\subsection*{3. Modélisation de l'Évolution des Régimes}

\textbf{Modèle utilisé :}
\[
\vec{R}_{t+1} = A \vec{R}_t + B \Delta Tech_t + C \Delta Legal_t + D \mathcal{I}_t
\]

\textbf{Simulation (simplifiée) :}
\begin{itemize}
	\item On suppose \(A = I\), \(B = [0.1, 0, 0, 0]\), \(C = [0, 0.1, 0, 0]\), \(D = [0, 0, 0.1, 0.1]\)
	\item Exemple : 
	\[
	\vec{R}_{2020} = (0.5, 0.2, 0.1, 0.2)
	\]
	\[
	\Delta Tech = 0.1, \Delta Legal = 0, \mathcal{I} = 0.1
	\]
	\[
	\vec{R}_{2030} \approx (0.6, 0.2, 0.2, 0.3)
	\]
	\item \textbf{Évolution future} : Renforcement de \(\alpha_T\) et \(\alpha_P\), affaiblissement de \(\alpha_J\)
\end{itemize}

\subsection*{4. Vérification de l'Accélération Technologique}

\textbf{Intervalles entre régimes :}
\begin{itemize}
	\item 1970-1990 : 20 ans
	\item 1990-2000 : 10 ans  
	\item 2000-2010 : 10 ans
	\item 2010-2020 : 10 ans
\end{itemize}

\textbf{Loi :} \(\Delta t_{n+1} = k \cdot \Delta t_n\)
\begin{itemize}
	\item \(k = \frac{10}{20} = 0.5\) entre 1970-1990 et 1990-2000
	\item Ensuite \(k = 1\) → pas d'accélération sur cette période
	\item \textbf{Prévision} : Si \(k = 0.8\), prochain changement vers \textbf{2028}
\end{itemize}

\subsection*{5. Analyse du Trilemme CRO Historique}

\textbf{Scores CRO estimés :}
\begin{itemize}
	\item \textbf{1970-1990} : C=0.3, R=0.5, O=0.2
	\item \textbf{1990-2000} : C=0.3, R=0.6, O=0.7
	\item \textbf{2000-2010} : C=0.5, R=0.8, O=0.8
	\item \textbf{2010-2020} : C=0.2, R=0.7, O=0.6
	\item \textbf{2020-...} : C=0.4, R=0.6, O=0.5
\end{itemize}

\textbf{Évolution :}
\begin{itemize}
	\item Amélioration de la \textbf{Fiabilité} et de l'\textbf{Opposabilité} jusqu'aux années 2010
	\item Puis baisse de la \textbf{Confidentialité} due au Big Data
	\item \textbf{Avenir} : Possible amélioration de C avec la cryptographie quantique
\end{itemize}

\section{Partie 3 : Investigation Historique Appliquée}

\subsection*{6. Reconstruction Archéologique d'Investigation}

\textbf{Affaire : Kevin Mitnick (1995)}
\begin{itemize}
	\item \textbf{Outils de l'époque} : Logs systèmes, traçage IP manuel, honeypots basiques
	\item \textbf{Analyse moderne} : IA pour corrélation de données, forensique cloud, analyse comportementale
	\item \textbf{Régimes de vérité} :
	\begin{itemize}
		\item \textbf{1995} : Juridique → preuve par chaîne de custody
		\item \textbf{Aujourd'hui} : Computational → preuve algorithmique
	\end{itemize}
	\item \textbf{Impact des limitations} : En 1995, la vérité était fragile, basée sur des preuves limitées et une expertise humaine
\end{itemize}

\subsection*{7. Projet de Recherche Archéologique}

\textbf{Trou identifié} : Manque d'études sur les pratiques investigatives avant l'Internet grand public (années 1980)

\textbf{Hypothèse} : Les premières investigations étaient influencées par les normes militaires et la sécurité physique

\textbf{Sources} : RFC, archives des premiers incidents (ex : 414s), publications techniques

\textbf{Méthode foucaldienne} :
\begin{itemize}
	\item Analyser les discours sur la preuve numérique dans les années 1980
	\item Identifier les formations discursives et le régime de vérité en émergence
	\item Cartographier les acteurs et institutions légitimes
\end{itemize}

\textbf{Article académique} :
\begin{itemize}
	\item Cadre : Foucault, Latour
	\item Méthode : Archéologie du savoir  
	\item Résultats : Mise en évidence d'un régime technique dominé par les experts
\end{itemize}

\subsection*{8. Analyse Prospective des Régimes Futurs}

\textbf{Scénario 2030-2050} : \textbf{Régime neuro-digital}
\begin{itemize}
	\item \textbf{Régime de vérité} : Preuves basées sur les signaux cérébraux (interfaces cerveau-machine)
	\item \textbf{Conditions de possibilité} :
	\begin{itemize}
		\item Avancées en neuroscience
		\item Cadres juridiques pour les données neuronales
	\end{itemize}
	\item \textbf{Méthodologie d'investigation} :
	\begin{itemize}
		\item Analyse des signaux cérébraux
		\item Reconstruction des intentions
		\item Validation par experts en neuro-éthique
	\end{itemize}
	\item \textbf{Défis} :
	\begin{itemize}
		\item Éthique : vie privée mentale, consentement
		\item Épistémologique : objectivité des preuves neuronales
	\end{itemize}
\end{itemize}

	
	
	
	
	
	
\end{document}
